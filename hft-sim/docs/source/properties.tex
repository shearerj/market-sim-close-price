\documentclass{article}

\usepackage{hyperref}
\usepackage{booktabs}
\usepackage{tabulary}

\newcommand{\email}[1]{\href{mailto:#1}{\texttt{#1}}}

\begin{document}
	
\title{Properties Files}
\author{
  Erik Brinkman \\
  \email{erik.brinkman@umich.edu}
  \and
  Elaine Wah \\
  \email{ewah@umich.edu}
}
\date{Updated: \today}
\maketitle

\section{Overview}

The properties file is a file that specifies details about the simulation that
don't affect the way that the simulation runs. This file is different than the
simulation spec file in that simulation spec changes actually affect the
numbers in the simulation. Properties files use the standard java properties
format which is in the \texttt{key = value} format. For fields that take a list
of values, the individual values are separated by semicolons (\texttt{;}).

\section{Keys}

\begin{table}
\centering
\begin{tabulary}{\textwidth}{lccL}\toprule
\textbf{Key} & \textbf{Default} & \textbf{EGTA Default} & \textbf{Description} \\ \midrule
\texttt{logLevel} & 0 & 0 & The type of logs to actually output. One of \{0, 1, 2, 3\},
where 0 is no logging, and 3 is debug logging. The is turned off by default,
because it's very slow.\\
\texttt{egta} & false & true & Whether this run is an egta run or not. egta
runs will only output player role, strategy, payoff information and nothing
else. \\
\texttt{whitelist} & & & A set of regexes that describe which features to
output. A feature name must match one of the regexes to be output in the final
observation. Multiple regexes can be concatenated by semicolons. To include all
features simply add the regex ``.*''. \\
\texttt{periods} & & & The periods at which to sample various periodic
information. \\
\bottomrule
\end{tabulary}
\caption{List of properties configuration keys.}
\label{tab:props}
\end{table}

\end{document}
