\documentclass[11pt]{article}
\usepackage[margin=1.2in]{geometry}
\usepackage[T1]{fontenc}
\usepackage{color,soul,hyperref,url}
\usepackage{graphicx,amstext}
\usepackage{subfigure}
\usepackage{amsmath}
\usepackage{enumerate}
\usepackage{verbatim}

\newcommand{\email}[1]{\href{mailto:#1}{\texttt{#1}}}

\title{Setup}
\author{
  Elaine Wah \\
  \email{ewah@umich.edu}
  \and
  Erik Brinkman \\
  \email{erik.brinkman@umich.edu}
}
\date{Updated: \today}

\begin{document}

\maketitle

\section{Initial Setup}

Before you get a copy of the code you need to make sure that you have the
following things installed on your machine. If you have any trouble installing
them contact me.

\begin{description}
\item[Open SSH:] To connect to the remote repository. You can verify you have it
  if \texttt{which ssh} returns something
\item[Git:] To get a copy of the source code. \verb|which git|
\item[Java 6:] We are using Java 6 for a number of reasons. If \verb|vava -version| says \verb|1.6| somewhere then you have Java 1.6 installed, but you may
  have it even if that says \verb|1.7| or \verb|1.8.|
\item[Ant:] To compile Java. If you have Java you probably have Apache Ant already.
\end{description}

\section{Connecting to the repository}

The repository for the HFT simulation system is hosted on \path{hft.eecs.umich.edu}. If you need an account on this machine in order to run simulations, please email \href{mailto:ewah@umich.edu}{Elaine}.
To access the repository, you will need to generate a public RSA key:
\begin{enumerate}
\item Generate a public/private RSA key pair with the following command:

\begin{verbatim}
ssh-keygen -t rsa -C "<youremailhere>"
\end{verbatim}

\item Press enter when prompted for the name to save under. Do not change the name from \path{id_RSA.pub}, and do not change the location where it is saved.
\item Enter a non-empty passphrase when prompted.
\item Email Elaine the \path{.ssh/id_RSA.pub} file located in your home directory.
\item After you receive notification that your key has been authorized to connect, proceed to the next step.
\item Create a directory to hold all of your HFT-related work, and then enter it by running the following (or similar) command:
\begin{verbatim}
cd
mkdir hft_project
cd hft_project
\end{verbatim}

\item Clone the git repository to get a copy of the source code
\begin{verbatim}
git clone git@hft.eecs.umich.edu:hft.git
\end{verbatim}

\end{enumerate}

\section{Setting up Eclipse}
It will be easiest to debug any new code added to the simulation system using an IDE. The following instructions are for Eclipse, but any other IDE for Java will suffice. Note that instructions are based on the Eclipse for Linux, and may differ depending on your OS. For the rest of the instructions we will assume that the source code is in \path{~/hft_project/hft/hft-sim}, which is the path for the HFT simulation system project. If you checked out the code somewhere else, make the necessary changes.

\begin{enumerate}
\item Download and unzip ``Eclipse for Java Developers'' (\href{http://www.eclipse.org/downloads/}{eclipse.org/downloads})
\item Start up Eclipse. For your workspace, select the path to
  the \path{hft} folder cloned from the repository (e.g. \path{~/hft_project/hft}).
\item Set Eclipse to auto-refresh the workspace
  \begin{enumerate}
  \item Select Window from the top bar and go down to Preferences
  \item Expand \verb|General| and then click on \verb|Workspace|
  \item Check \verb|Refresh using native hooks or polling| and \verb|Refresh on access|
  \item Click OK
  \end{enumerate}
  
\item From the File menu, select New >> Project...
\item Select \verb|Java Project from Existing Ant Buildfile|.
\item Browse for the Ant buildfile; select the \path{build.xml} file located in \path{hft-sim}, the simulation system directory
  (\path{~/hft_project/hft/hft-sim/build.xml}). Leave the project name unchanged and
  click Finish.
\item At this point, the project will be imported automatically.
\item Fix the project build path:
  \begin{enumerate}
  \item Right click on the \verb|hft-sim| project and select Properties.
  \item Click Java Build Path in the left panel, then select Libraries (at the
    top). Remove the JRE\_LIB Library.
  \item Click \verb|Add Library| and add a JRE System Library (make sure you add a Java 6 library!).
  \item Select Source at the top (next to libraries).
  \item Click Add Folder and select \verb|test|.
  \item Select \verb|Output folder| under hft-sim/src and then click edit.
  \item Select \verb|Project's default output folder| and then OK to close the window.
  \item Click OK and close the window.
  \end{enumerate}

\item Before getting setup to run the simulator, you need to create a directory in which to run your simulations.  To keep the repository organized, run all of your simulations within the \path{~/hft_project/hft/hft-sim/simulations} directory.
  \begin{enumerate}
  \item Create a folder in the \path{hft-sim} folder called \path{simulations}.
  \item Create a folder within simulations (e.g. \path{simulations/test}).
  \item Copy \path{docs/simulation_spec.json} into the folder you just created.
  \end{enumerate}

\item In the Properties window set up Run/Debug configurations:
  \begin{enumerate}
  \item Right click on the project again and select Properties.
  \item Select \verb|Run/Debug Settings| in the left panel. Add a new launch configuration (Java Application).
  \item At the top, change the name to \verb|Run Simulator.| To select the
    correct main class, click \verb|Search...| and select \verb|SystemManager|.
  \item Select the Arguments tab and enter something like the following into the \verb|Program Arguments| window:

\begin{verbatim}
simulations/test 1
\end{verbatim}

The first argument is the path to the simulation directory that you created above.  This can be any directory as long as it has a
\path{simulation_spec.json} file in it.

The second argument represents what observation number to use. In this case we're specifying run 1, but it can be any arbitrary number. This number is appended to the name of generated observation files.
  \end{enumerate}
  
\item Save your settings and you should now be able to run the simulator in Eclipse!

\item Test that everything worked:
  \begin{enumerate}
  \item Click the down arrow next to the green arrow Run icon and select
    \verb|Run Configurations...|
  \item Select \verb|Run Simulator| from the left hand column and click Run.
  \item Check that there's an observation file called
    \path{simulations/test/observation1.json}, and a log file in
    \path{simulations/test/logs}.
  \end{enumerate}

\end{enumerate}

\section{Setting up EclEmma Code Coverage Tool in Eclipse}

To install EclEmma, you first need to have the Eclipse Marketplace Client
installed. If you see \verb|Eclipse Marketplace...| under the \verb|Help| menu,
then the Eclipse Marketplace Client is installed, and you can skip to section
\ref{ssec:eclemma}.

\subsection{Installing the Eclipse Marketplace Client}

\begin{enumerate}
\item Go to \verb|Help > Install New Software...|
\item Click the drop down arrow next to \verb|Work with:| and select
  \verb|--All Available Sites--|
\item In the search box that says \verb|type filter text| enter
  \verb|marketplace|. This may lag a lot, eclipse is just poorly threaded
\item You should see a listing called \verb|Marketplace Client|. Select the
  checkbox next to it, and then click \verb|Next >| at the bottom
\item The next page should that you are installing the Marketplace
  Client. Click \verb|Next >| again
\item Read and accept the license agreement. Then click \verb|Finish|. Eclipse
  should begin downloading the marketplace
\item After the Marketplace client is installed, Eclipse should prompt you to
  restart Eclipse. When you're ready, click \verb|Yes|
\end{enumerate}

\subsection{Installing EclEmma}\label{ssec:eclemma}

\begin{enumerate}
\item Go to \verb|Help > Eclipse Marketplace...|
\item Type \verb|eclemma| in the searchbox next to \verb|Find:| and hit
  \verb|Return|
\item You should see a single entry called \verb|EclEmma Java Code Coverage|.
  Click the install button next to it
\item You should be prompted with a confirm dialogue. Click \verb|Next >|
\item Read and accept the license agreement. Then click \verb|Finish|
\item After installation, you should be prompted to restart Eclipse. When
  you're ready, click \verb|Yes|
\end{enumerate}

\subsection{Using EclEmma}

Traditionally with Eclipse, you can right click on a JUnit test, or a package
with JUnit tests in it, and select \verb|Run As > JUnit Test|. Once EclEmma is
installed, if you right click on a JUnit test, or a package containing JUnit
tests, you can go to \verb|Coverage As > JUnit Test|. This will run all of the
JUnit tests and show code coverage.

After running coverage, there are two main sources of coverage information. You
should see a View called \verb|Coverage| which lists the coverage of every
folder in the project. In addition, each line in Eclipse will be highlighted
Green, Yellow, or Red indicating Covered, Partially Covered, or Not Covered
respectively.

To remove the line highlighting from your source and test files, simple click
the black \verb|X| icon in the coverage View called \verb|Remove Active Session|.

\end{document}
