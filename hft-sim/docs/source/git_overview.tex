\documentclass[11pt]{article}
\usepackage[margin=.9in]{geometry}
\usepackage{color,soul,hyperref,url}
\usepackage{graphicx,amstext}
\usepackage{subfigure}
\usepackage{amsmath}
\usepackage{enumerate}
\usepackage{booktabs}

\newcommand{\email}[1]{\href{mailto:#1}{\texttt{#1}}}

\begin{document}
	
\title{Git Overview}
\author{
  Elaine Wah \\
  \email{ewah@umich.edu}
}
\date{Updated: May 2013}
\maketitle

% ---------------------
\section{Useful git commands}

\textsc{Remote} refers to the repository. \textsc{Local} refers to a file or branch on your machine. Note that local branches will not be added to the repository unless they are pushed. To work with a remote-tracking branch, you can pull from it to a local branch.

Please do not commit anything other than source code (i.e., do not commit any observation or CSV files generated from simulations). Also, do not leave the commit message empty.\\

\begin{center}
  \begin{tabular}{ll} \toprule
    %Description & Command \\  \hline
    \multicolumn{2}{c}{Branching} \\ \midrule
    List all (local) branches & \texttt{git branch} \\
    List all (local + remote) branches & \texttt{git branch -a} \\
    Check out (switch to) a (local) branch & \texttt{git checkout <my branch>} \\
    Create/checkout a new (local) branch & \texttt{git checkout -b <new branch name>} \\
    Create/checkout branch to track remote one & \texttt{git checkout -t origin/<remote branch>} \\
    Update list of all branches & \texttt{git fetch} \\
    Merge from local branch to checked-out branch & \texttt{git merge <branch to merge from>} \\
    \\
    \toprule
    \multicolumn{2}{c}{Pulling} \\ \midrule
    Pull changes from master branch on remote and & \texttt{git pull origin master} \\
    merge into checked-out branch \\
    Pull from a branch on the remote & \texttt{git pull origin <remote branch>} \\
    \\
    \toprule
    \multicolumn{2}{c}{Adding/removing/committing} \\ \midrule
    Check status of checked-out branch & \texttt{git status} \\
    Add a file or folder to be commited & \texttt{git add <file/folder name>} \\
    Remove a file & \texttt{git rm <file name>} \\
    Commit a file to the checked-out branch & \texttt{git commit -m "<commit
      message>"} \\
    & (\texttt{git commit} will open an editor) \\
    Push a branch to the repository & \texttt{git push origin <my branch>} \\
    \\
    \toprule
    \multicolumn{2}{c}{Miscellaneous} \\ \midrule
    Save working directory without committing & \texttt{git stash} \\
    (useful when wish to switch branches) \\
    Continue from where stashed & \texttt{git stash pop} \\
    \bottomrule
  \end{tabular}
\end{center}

\end{document}
