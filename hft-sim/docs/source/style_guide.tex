\documentclass{article}

\usepackage{hyperref}

\newcommand{\email}[1]{\href{mailto:#1}{\texttt{#1}}}

\begin{document}
\title{Style Guide}
\author{Erik Brinkman \\ \email{erik.brinkman@umich.edu}}
\maketitle

\section{Outline}

\textbf{Eclipse Format File?}

\begin{enumerate}
\item Test class interface, not implementation
\item Tests should be as atmoic as possible
\item Iterables whenever possible, guava helps
\item Never wrap objects like double for comparison, use the primitive type
\item Keep class tests in appropriate class, don't duplicate classes
\item Don't repeat code, if code is used frequently abstract it to a function,
  tests or otherwise
\item If statements always on a new line
\item Use Guava over not using guava (Optional, Ordering, Collections,
  ImmutableCollections)
\item Don't break interface specifications, e.g. assuming an order to a
  collection. Even if there is one, it's not part of the interface, and may
  change
\item Create a setup method than can be tweaked to properly set up any
  environment in testing
\item Eclipse format file?
\item Factory methods and protected constructors
\item make proper member variables final
\end{enumerate}

\section{Copied from Old Google Doc}

Best practices
Development
Never wrap objects like double for comparison, use the primitive type
Don’t repeat code, if code is used frequently abstract it to a function, tests
or otherwise
Use guava over not using guava (Optional, Ordering, Collections)
Iterables whenever possible, guava helps
Don’t break interface specifications, e.g. assuming an order to a
collection. Even if there is one, it’s not part of the interface, and may
change
May introduce a format file, but still in the works
Factory methods and protected constructors
Make proper member variables final or private and provide methods if necessary.
Try not doing if type == BUY { // stuff } else { // stuff } because it's very
error prone. Instead try to do type.sign() * stuff
Use final only when it matters (for class variables or variables that are
referenced in anonymous classes, or member variables that would never be
written to
assert are good. Run with -ea, always enabled on testing
Efficiency
Never write anything that scales linearly with the number of time steps
Avoid building large strings that may or may not be used (e.g. logging) most
utilities should provide for doing this last minute
Testing
Create a setup method than can be tweaked to properly set up any environment in
testing
Keep class tests in appropriate class, don’t duplicate classes
Test class interface, not implementation
Tests should be as atomic as possible
EclEmma is a nice eclipse extension that will test coverage. Try to make sure
new code is 100\% covered, but don't just aim for coverage as 100\% coverage can still miss bugs
Working with repository
If you're adding new commits to master, rebase instead of merge to keep our
versioning more consistent
Tag simulator versions uploaded to the testbed
Style guide
if statements always on a new line
Match existing style
Naming parameters
Strategy parameters - the shorter the better (document appropriately)
Environment parameters - can be more descriptive, must match name in Keys class
Other suggestions?

\end{document}
