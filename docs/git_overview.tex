\documentclass[11pt]{article}
\usepackage[margin=1.2in]{geometry}
\usepackage{color,soul,hyperref,url}
\usepackage{graphicx,amstext}
\usepackage{subfigure}
\usepackage{amsmath}
\usepackage{enumerate}
\def\newblock{}

\newcommand{\bid}{\ensuremath{\mathit BID}}
\newcommand{\ask}{\ensuremath{\mathit ASK}}
\newcommand{\PV}{\ensuremath{\mathit PV}}
\providecommand{\floor}[1]{\lfloor#1\rfloor}
\providecommand{\ceil}[1]{\lceil#1\rceil}

\begin{document}
	
\title{Git Overview}
%\numberofauthors{1}
\author{
%\alignauthor
Elaine Wah\\
\href{mailto:ewah@umich.edu}{ewah@umich.edu}
}
\date{Updated: May 2013}
\maketitle


% ---------------------
\section{Useful git commands}

\textsc{Remote} refers to the repository. \textsc{Local} refers to a file or branch on your machine. Note that local branches will not be added to the repository unless they are pushed. To work with a remote-tracking branch, you can pull from it to a local branch.

Please do not commit anything other than source code (i.e., do not commit any observation or CSV files generated from simulations). Also, do not leave the commit message empty.\\

\noindent\begin{tabular}[t]{p{0.5\textwidth} p{0.5\textwidth}} \hline
 	%Description & Command \\  \hline
	\multicolumn{2}{c}{Branching} \\ \hline
	List all (local) branches						& \verb|git branch| \\
	List all (local + remote) branches 				& \verb|git branch -a| \\
	Check out (switch to) a (local) branch						& \verb|git checkout <my branch>| \\
	Create/checkout a new (local) branch				& \verb|git checkout -b <new branch name>| \\
	Create/checkout branch to track remote one & \verb|git checkout -t origin/<remote branch>| \\
	Update list of all branches						& \verb|git fetch| \\
	Merge from local branch to checked-out branch 	& \verb|git merge <branch to merge from>| \\
	
	\\ \hline
	\multicolumn{2}{c}{Pulling} \\ \hline
	Pull changes from master branch on remote and merge into checked-out branch		& \verb|git pull origin master| \\
	Pull from a branch on the remote 			& \verb|git pull origin <remote branch>| \\

	\\ \hline
	\multicolumn{2}{c}{Adding/removing/committing} \\ \hline
	Check status of checked-out branch		& \verb|git status| \\
	Add a file or folder to be commited		& \verb|git add <file/folder name>| \\
	Remove a file	& \verb|git rm <file name>| \\
	Commit a file to the checked-out branch	& \verb|git commit -m "<commit message>"| (\verb|git commit| will open an editor) \\
	Push a branch to the repository  			&\verb|git push origin <my branch>| \\
	
	\\ \hline
	\multicolumn{2}{c}{Miscellaneous} \\ \hline
	Save working directory without committing (useful when wish to switch branches) & \verb|git stash| \\
	Continue from where stashed								& \verb|git stash pop|
\end{tabular}


\end{document}






