\documentclass[11pt]{article}
\usepackage[margin=1.2in]{geometry}
\usepackage{color,soul,hyperref,url}
\usepackage{graphicx,amstext}
\usepackage{subfigure}
\usepackage{amsmath}
\usepackage{booktabs}
\usepackage{listings}
\usepackage{paralist}

\lstset{
  basicstyle=\ttfamily,
  breaklines=true
}

\title{Simulation System Overview}
\author{
  Elaine Wah \\
  \href{mailto:ewah@umich.edu}{ewah@umich.edu}
}
\date{Updated: \today}

\begin{document}

\maketitle

\section{Introduction}

The HFT simulation system employs agent-based modeling and discrete-event simulation to model algorithmic trading in financial markets.

\section{Overview}

The simulation system consists of:
\begin{description}
\item[Markets:] Two types of markets are implemented in the system, a
  continuous double auction (CDA) market and a call market (which matches orders at regular, fixed
  intervals).

\item[Agents:] There are three general types (i.e. roles) of agents in the system,
  which are distinguished by whether or not they have fast access to more than
  one market and whether or not they possess private valuations for the security.

\begin{description}
\item[Background traders:] These agents have (potentially
  undelayed) access to only a single market (which is specified at agent creation as their primary market). See
  \href{../src/entity/agent/ZIAgent.java}{\textsf{ZIAgent}}, \href{../src/entity/agent/ZIRAgent.java}{\textsf{ZIRAgent}}, \href{../src/entity/agent/ZIPAgent.java}{\textsf{ZIPAgent}}, and \href{../src/entity/agent/AAAgent.java}{\textsf{AAAgent}}.

\item[Market makers (MM):] These agents submit a ladder of buy and sell orders upon each reentry. See
  \href{../src/entity/agent/BasicMarketMaker.java}{\textsf{BasicMM}},
  \href{../src/entity/agent/MAMarketMaker.java}{\textsf{MAMM}}, and
  \href{../src/entity/agent/WMAMarketMaker.java}{\textsf{WMAMM}}.

\item[High-frequency traders (HFT):] These agents have access to
  multiple markets (usually every market), which models the
  direct feeds that many HFTs have to exchanges. See \href{../src/entity/agent/LAAgent.java}{\textsf{LAAgent}}.
\end{description}
\end{description}

\section{Discrete-Event Simulation}

In our system, we employ \emph{discrete-event simulation}, a paradigm that
allows the precise specification of event occurrences. It is particularly
effective for modeling current U.S. securities regulation, specifically
Regulation NMS---which led to the creation of the Security Information Processor
(SIP) and which mandates the routing of orders for best execution.  Components
of the simulation system include:

\begin{description}
\item[Entity:] Objects present in the simulation system, e.g., traders, markets, quote processors (QP), transaction processors (TP), and the SIP, that perform actions that effects on other entities.

\item[Activity:] Actions that entities can execute.

\item[Event:] A sequence of activities happening at the same time. Maintains the
  order in which they occur.

\item[Event Queue:] Queue ordered by activity time that executes activities in
  the ``proper'' order, sequentially until empty. Multiple activities may occur
  during the same time step, and in most circumstances the order they are
  executed in is pseudo-random according to the random number generator of the
  simulation. The random nature is meant to simulate the slight timing
  differences that occur in real life (nothing actually occurs at the same
  time).
\end{description}

To summarize, an \emph{event} consists of a sequence of \emph{activities} that
are to be executed by various \emph{entities} (traders, markets, and the SIP).


\subsection{Activities}

Each activity has a timestamp and each is associated with at least one entity
present in the simulation system. Note that activities may be chained (the next
immediate activity is inserted at the end of the current one). See
Table~\ref{tab:activity} for a partial list of activities within the system.

\begin{table}
\centering
\begin{tabular}{l l} \toprule
\textbf{Activity} 	& \textbf{Description} \\ \midrule
\textsf{AgentArrival} 	& agent arrives in a market (or markets) \\
\textsf{AgentStrategy} 	& agent executes its trading strategy \\
\textsf{Clear} 		& market clears any matching orders \\
\textsf{Liquidate}	& agent liquidates any net position \\
\textsf{LiquidateAtFundamental}   & agent liquidates inventory at fundamental value \\
\textsf{ProcessQuote} 	& QP (or SIP) updates its best market quotes \\
\textsf{ProcessTransactions}  & TP (or SIP) updates its list of transactions \\
\textsf{SendToQP} 	& market sends a new quote to QP (or SIP) \\
\textsf{SendToTP}   & market sends list of new transactions to TP (or SIP) \\
\textsf{SubmitOrder} 	& agent submits a limit order to a market \\
\textsf{SubmitNMSOrder} 	& agent submits a limit order, routed for best execution \\
\textsf{WithdrawOrder} 	& agent withdraws a specific order from a market \\ \bottomrule
\end{tabular}
\caption{List of activities in the simulation system.}
\label{tab:activity}
\end{table}

\subsection{Example}

To control the latency of the SIP as well as general market access, we specify
two activities: \textsf{SendToIP} and \textsf{ProcessQuote}.  The
\textsf{SendToIP} activity is inserted when a market updates its quote at time
$t$. Once the information processor (IP) gets the information it inserts a
\textsf{ProcessQuote} activity to execute at time $t + \delta$ in the future to
account for the delay caused by processing the information.
%
When \textsf{ProcessQuote} is executed, the IP updates its stored information on
the best market quotes. When agent's query the SIP for market information they
will only get the most recent information that it's processed after $t+\delta$,
not all of the market quotes at the current time.

\section{Simulation Specification File}

See the \path{simulation_spec.pdf} file for details on specifying the simulation and environment parameters.

\section{Running a Simulation}

\begin{enumerate}
\item To run a basic simulation, create a directory to store all of your
  simulations (e.g. \path{simulations}) and then create a directory for this
  specific simulation inside it:

\begin{verbatim}
pwd # should print your hft folder
mkdir simulations
mkdir simulations/test
\end{verbatim}

\item Then copy the default \path{simulation_spec.json} file into the folder you
  just created:

\begin{verbatim}
cp docs/simulation_spec.json simulations/test/
\end{verbatim}

\item Make any tweaks to the specification you want using your favorite text
  editor.

\item Use the following command to run your simulation:

\begin{verbatim}
./run_hft.sh simulations/test <number of simulations to run>
\end{verbatim}

For example:

\begin{verbatim}
./run_hft.sh simulations/test 100
\end{verbatim}

\item All simulations should be saved within subfolders in the \path{simulations/test}
(or other) directory!

\end{enumerate}

\end{document}
