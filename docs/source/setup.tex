\documentclass[11pt]{article}
\usepackage[margin=1.2in]{geometry}
\usepackage[T1]{fontenc}
\usepackage{color,soul,hyperref,url}
\usepackage{graphicx,amstext}
\usepackage{subfigure}
\usepackage{amsmath}
\usepackage{enumerate}

\title{Setup}
\author{
  Elaine Wah \\
  \href{mailto:ewah@umich.edu}{ewah@umich.edu}
}
\date{Updated: \today}

\begin{document}

\maketitle

\section{Initial Setup}

Before you get a copy of the code you need to make sure that you have the
following things installed on your machine. If you have any trouble installing
them contact me.

\begin{description}
\item[Open SSH:] To connect to the remote repository. You can verify you have it
  if \texttt{which ssh} returns something
\item[Git:] To get a copy of the source code. \texttt{which git}
\item[Java 6:] We are using java 6 for a number of reasons. If \texttt{java
  -version} says ``1.6'' somewhere then you have java 1.6 installed, but you may
  have it even if that says ``1.7'' or ``1.8.''
\item[Ant:] To compile java. If you have java you probably have ant.
\end{description}

\section{Connecting to the repository}

The repository for the HFT simulation system is hosted on \path{hft.eecs.umich.edu}. If
you
need an account on this machine in order to run simulations, please email me.
To access the repository, you will need to generate a public RSA key:
\begin{enumerate}
\item Generate a public/private rsa key pair with the following command:

\begin{verbatim}
ssh-keygen -t rsa -C "<youremailhere>"
\end{verbatim}

\item Press enter when prompted for the name to save under. Do not change the
  name from \path{id_rsa.pub}, and do not change the location where it is saved.
\item Enter a non-empty passphrase when prompted.
\item Email me the \path{.ssh/id_rsa.pub} file located in your home directory.
\item I will need to authorize your key to connect. After I've notified you to
  let you know this is done, proceed to the next step.
\item Create a directory to hold all of your hft work, and then enter it by
  running the following (or similar) command:

\begin{verbatim}
cd
mkdir hft_project
cd hft_project
\end{verbatim}

\item Clone the git repository to get a copy of the source code

\begin{verbatim}
git clone git@hft.eecs.umich.edu:hft.git
\end{verbatim}

\item Move to the repository directory and switch to the reorg branch

\begin{verbatim}
cd hft
git checkout reorg
\end{verbatim}

\end{enumerate}

\section{Setting up Eclipse}
It will be easiest to debug any new code added to the simulation system using an
IDE. The following instructions are for Eclipse, but any other IDE for Java will
suffice. Note that instructions are based on the Eclipse for Linux, and may
differ depending on your OS. For the rest of the instructions we will assume
that the source code is in \path{~/hft_project/hft}. If you checked out the code
somewhere else, make the necessary changes.

\begin{enumerate}
\item Download and unzip ``Eclipse for Java Developers'' (\href{http://www.eclipse.org/downloads/}{eclipse.org/downloads})
\item Start up Eclipse. For your workspace, select the directory that contains
  the \path{hft} folder cloned from the repository (e.g. \path{~/hft_project}).
\item Set Eclipse to auto-refresh the workspace
  \begin{enumerate}
  \item Select Window from the top bar and go down to Preferences
  \item Expand ``General'' and then click on ``Workspace''
  \item Check ``Refresh using native hooks or polling'' and ``Refresh on access''
  \item Click OK
  \end{enumerate}
  
\item From the File menu, select New Project.
\item Select ``Java Project from Existing Ant Buildfile.''
\item Browse for the Ant buildfile; select the \path{build.xml} file located in
  the top level directory of the repository
  (\path{~/hft_project/hft/build.xml}). Leave the project name unchanged and
  click Finish.
\item At this point, the project will be imported automatically.
\item Fix the project build path:
  \begin{enumerate}
  \item Click File >> Properties.
  \item Click Java Build Path in the left panel, then select Libraries (at the
    top). Remove the JRE System Library.
  \item Click ``Add Library'' and add a JRE System Library (make sure you add a
    java 6 library!).
  \item Select Source at the top (next to libraries).
  \item Click Add Folder and select ``test.''
  \item Click OK and close the window.
  \end{enumerate}

\item Before getting setup to run the simulator, you need to create a directory
  to run your simulations in.  To keep the repository organized, run all of your
  simulations within the \path{simulations} directory.
  \begin{enumerate}
  \item Created a folder in the root of the repository called \path{simulations}.
  \item Create a folder within simulations (e.g. \path{simulations/test}).
  \item Copy \path{docs/simulation_spec.json} into the folder you just created.
  \end{enumerate}

\item In the Properties window setup run/debug configurations:
  \begin{enumerate}
  \item Right click on the project again and select Properties.
  \item Select Run/Debug Settings in the left panel. Add a new launch
    configuration (Java Application).
  \item At the top, change the name to ``Run Simulator.'' To select the
    correct main class, click ``Search...'' and select SystemManager.
  \item Select the Arguments tab and enter something like the following into the
    ``Program Arguments'' window:

\begin{verbatim}
simulations/test 1
\end{verbatim}

The first argument is the path to the simulation directory that you created
above.  This can be any directory as long as it has a
\path{simulation_spec.json} file in it.

The second argument represents what run number to use. In this case we're
specifying run 1, but it can be any arbitrary number. This number is appended
to the name of generated observation files.
  \end{enumerate}
  
\item Save your settings and you should now be able to run the simulator in
Eclipse!

\item Test that everything worked:
  \begin{enumerate}
  \item Click the down arrow next to the green arrow Run icon and select ``Run
    Configurations...''
  \item Select ``Run Simulator'' from the left hand column and click Run
  \item Check that there's an observation file called
    \path{simulations/test/observation1.json}, and a log file in
    \path{simulations/test/logs}.
  \end{enumerate}

\end{enumerate}

\end{document}
