\documentclass[11pt]{article}
\usepackage[margin=1.2in]{geometry}
\usepackage{color,soul,hyperref,url}
\usepackage{graphicx,amstext}
\usepackage{subfigure}
\usepackage{amsmath}
\usepackage{enumerate}
\usepackage{listings}
\usepackage{paralist}

\lstset{
  basicstyle=\ttfamily,
  breaklines=true
}

\begin{document}
	
\title{How To}
\author{
  Erik Brinkman \\
  \href{mailto:erik.brinkman@umich.edu}{erik.brinkman@umich.edu}
  \and
  Elaine Wah \\
  \href{mailto:ewah@umich.edu}{ewah@umich.edu}
}
\date{Updated: \today}
\maketitle

\tableofcontents

\section{Add a New Agent}

\begin{enumerate}[1)]
\item To add a new Agent, first create a class for it in the
  \texttt{entity.agent} package. A new agent will have to do a few things:
  \begin{enumerate}[1.]
  \item Extend an existing Agent class. There is an extensive hierarchy, but the
    most common extensible Agents are likely BackgroundAgent, MarketMaker,
    HFTAgent, or a subclass of one of those.
  \item Create two constructors. One constructor will parse an EntityProperties
    object to determine what various properties to set. This constructor will
    then call the other one with the hard coded values. See ZIRAgent for a
    reference.
  \item Create a unique strategy for itself. A strategy looks at all available
    information to the agent, which are usually stored in internal variables or
    the Agents corresponding InformationProcessors, and then returns a List of
    Activities to be scheduled. ZIRAgent will serve a reference point. By
    default, all BackgroundAgents have the ability to act again after a random
    amount of time. If you wish to have your agent do so, simply add all of
    \texttt{super.agentStrategy(currentTime))} to your list of activities if you
    also want your agent to act again according to standard settings.
  \end{enumerate}
\item Add a new \texttt{AgentType} to \texttt{systemmanager.Consts}. This will
  allow people to refer to your agent. This will function as the Agent's name.
\item Add a new entry in the switch statement inside
  \texttt{entity.agent.AgentFactory}. Here you should write code that will
  construct your agent type with the given parameters. The existing fragments
  inside AgentFactory should be enough to go off of.
\end{enumerate}

You should now have a functioning agent! In order to add some to the simulation,
simply add a line in the \path{simulation_spec.json} file with a key that
matches the name of its \texttt{AgentType}, and a value that contains the
parameters for your Agent. It may be helpful to read about adding a new
parameter in Section~\ref{sec:new_param}, or reading about the simulation spec
file in. % TODO Reference the file

\section{Add a Unit Test}

\section{Add a New Parameter} \label{sec:new_param}

\section{Add a Preset Market Model}

\section{Add a New Statistic}

\end{document}
