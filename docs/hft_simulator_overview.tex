\documentclass[11pt]{article}
\usepackage[margin=1.2in]{geometry}
\usepackage{color,soul,hyperref,url}
\usepackage{graphicx,amstext}
\usepackage{subfigure}
\usepackage{amsmath}
\def\newblock{}

\newcommand{\bid}{\ensuremath{\mathit BID}}
\newcommand{\ask}{\ensuremath{\mathit ASK}}
\newcommand{\PV}{\ensuremath{\mathit PV}}
\providecommand{\floor}[1]{\lfloor#1\rfloor}
\providecommand{\ceil}[1]{\lceil#1\rceil}

\begin{document}
	
\title{HFT Simulation System \\
Part I: Overview}
%\numberofauthors{1}
\author{
%\alignauthor
Elaine Wah\\
\href{mailto:ewah@umich.edu}{ewah@umich.edu}
}
\date{Updated: May 2013}
\maketitle


% ---------------------
\section{Introduction}

The HFT simulation system employs agent-based modeling and discrete-event simulation to model communication latencies and current U.S. securities regulations.

%TODO
%To run simulations and access the repository, you will need 1) Java, 2) Apache Ant, 3) Git.

% ---------------------
\section{Overview}

The simulation system consists of:
\begin{itemize}
\item \textsc{Market models}: Each model specifies the number, type of, and configuration parameters of markets present. The HFT simulation system allows the simulation of multiple market configurations in parallel.

For example, the \verb|TwoMarket| model specifies two Continuous Double Auction (CDA) markets, and the \verb|CentralCall| model specifies a single call market that clears at the same frequency as the NBBO update latency. 

\item \textsc{Markets}: Two types of markets are implemented in the system, a CDA market and a call market (which matches orders at regular, fixed intervals).

\item \textsc{Agents}: There are two general types of agents in the system, which are distinguished by whether or not they have direct access to more than one market. Note that agents are duplicated (with common random number seeds) in each market model, but they execute their strategy within each model independently.

\begin{itemize}
\item \textsc{Background agents} (\verb|BackgroundAgent|): These agents have direct, undelayed access to only a single market (which is specified at agent creation). Example: \verb|ZIAgent|.

\item \textsc{High-frequency trading agents} (\verb|HFTAgent|): These agents have access to multiple markets (all markets within a given market model), which models the direct feeds that many HFTs have to exchanges. Example: \verb|LAAgent|.
\end{itemize}
\end{itemize}


% ---------------------
\section{Discrete-event simulation}

In our system, we employ \emph{discrete-event simulation}, a paradigm that allows the precise specification of event occurrences. It is particularly effective for modeling current U.S. securities regulation, specifically Regulation NMS---which led to the creation of the Security Information Processor (SIP) and which mandates the routing of orders for best execution.
Components of the simulation system include:

%simulating the interactions between traders in a stock market.
\begin{itemize}
\item \textbf{Entity}: Objects present in the simulation system, e.g., traders, markets, and the SIP.

\item \textbf{Activity}: Actions that entities can execute.

\item \textbf{Event}: A sequence of activities, ordered by priority. If there is a tie, activities are inserted in order of arrival.
Priorities are assigned based on activity type (e.g., bid submission, market clearing).

\item \textbf{Event Queue}: Priority queue ordered by event time; executed sequentially until empty. Multiple events may occur during the same time step, but they are executed in the order in which they are enqueued.

\end{itemize}
%
To summarize, an \emph{event} consists of a sequence of \emph{activities} that are to be executed by various \emph{entities} (traders, markets, and the SIP).


\subsection{Activities}

Each activity has a timestamp and each is associated with at least one entity present in the simulation system. Note that activities are chained (the next immediate activity is inserted at the end of the current one). See Table~\ref{tab:activity} for a partial list of activities within the system.

\begin{table}
\centering
\begin{tabular}{l l}
\uppercase{Activity} 	& \uppercase{Description} \\ \hline
\textsf{AgentArrival} 	& agent arrives in a market (or markets) \\
\textsf{AgentReentry}	& agent re-enters a market (or markets) \\
\textsf{AgentStrategy} 	& agent executes its trading strategy \\
\textsf{Clear} 			& market clears any matching orders \\
\textsf{Liquidate} 		& agent liquidates any net position \\
\textsf{ProcessQuote} 	& SIP updates its best market quotes \\
\textsf{SendToSIP} 		& market sends a new quote to the SIP \\
\textsf{SubmitBid} 		& agent submits single-point bid to a market \\
\textsf{SubmitMultipleBid} 	& agent submits multi-point bid to a market \\
\textsf{SubmitNMSBid} 	& agent submits single-point bid, routed for best execution \\
\textsf{UpdateAllQuotes}	& agent updates its quotes for its market(s) \\
\textsf{UpdateNBBO} 		& SIP computes and publishes updated NBBO quote \\
\textsf{WithdrawBid} 	& agent withdraws its bid from a market \\
\end{tabular}
\caption{List of activities in the simulation system.}
\label{tab:activity}
\end{table}


\subsection{Example}

To control the latency of the SIP, we specify three activities: \verb|SendToSIP|, \verb|ProcessQuote|, and \verb|UpdateNBBO|.
The \verb|SendToSIP| activity is inserted when a market publishes a quote at time $t$; upon execution the market sends its updated quote to the SIP entity and inserts a \verb|ProcessQuote| and a \verb|UpdateNBBO| activity, both to execute at time $t + \delta$ in the future.
%
When \verb|ProcessQuote| is executed, the SIP updates its stored information on the best market quotes. It computes and publishes an updated NBBO based on this information during the execution of the \verb|UpdateNBBO| activity.

Figure~\ref{fig:simsystem} illustrates how the activities are sequenced in our simulation system to reflect the communication latencies arising as a consequence of market fragmentation.
Market 1 clears and publishes an updated quote at time $t_1$. Market 2 publishes its new quote at time $t_2$. For $\delta > t_2 - t_1$, a \verb|ProcessQuote| followed by an \verb|UpdateNBBO| activity are executed sequentially at $t_1 + \delta$, as well as at $t_2 + \delta$. The \verb|UpdateNBBO| executing at $t_1+\delta$ will not incorporate market 2's updated quote, as the \verb|ProcessQuote| activity to add market 2's best quote $(\bid_{2} , \ask_{2})$ will not be executed until $t_2 + \delta$.
This process serves to approximate the behavior of the SIP with a delay of $\delta$.

\begin{figure}
  \centering
  \includegraphics[width=0.75\textwidth]{sip_simulation_example}
  \caption{Event queue during the dissemination and processing of updated market quotes for NBBO computation, given latency $\delta > t_2 - t_1$.
  %The top of the event queue corresponds to the current time in the simulation.
  There are two markets, $M_1$ and $M_2$. When the NBBO update activity executes at time $t_1 + \delta$, the SIP has just processed the best quote $(\bid_{1} , \ask_{1})$ at time $t_1$ from market 1; this is therefore the most up-to-date information that could be reflected in the NBBO at time $t_1 + \delta$.}
  \label{fig:simsystem}
\end{figure}

% ---------------------
\section{Use cases}

There are two use cases for the simulation system:

\begin{itemize}
\item \textbf{EGTA}: In this use case, only a single market model may be specified.

\item \textbf{Market models}:
In the non-EGTA use case, you can simulate multiple market models, with identical background agent populations initialized across all models.

To specify an agent who is \emph{not} active in all market models, you will need to create a market model configuration that creates such an agent. The agent parameterization can be specified in the configuration string.

\end{itemize}


% ---------------------
\section{Simulation specification file}

The parameters in the specification file are those that can be adjusted in simulations. Below is an example \verb|simulation_spec.json| file that specifies a simulation configuration:

\begin{verbatim}
{
  "assignment": {
    "MARKETMAKER" [
      "BASICMM:sleepTime_200_numRungs_10_rungSize_1000"
    ],
    "BACKGROUND": [
      "ZIR:bidRange_2000_maxqty_5",
      "ZIR:bidRange_3000_maxqty_5"
    ]
  },
  "configuration": {
    "sim_length": "15000",
    "tick_size": "1",
    "primary_model": "CENTRALCDA",
    "TWOMARKET": "LA:sleepTime_0_alpha_0.001,DUMMY",
    "CENTRALCDA": "1",
    "CENTRALCALL": "NBBO",
    "BASICMM": "0",
    "BASICMM_setup": "sleepTime_200_numRungs_10_rungSize_1000",
    "ZI": "3",
    "ZI_setup": "bidRange_2000",
    "ZIR": "0",
    "ZIR_setup": "bidRange_2000_maxqty_5"
    "ZIP": "0",
    "ZIP_setup": "",
    "nbbo_latency": "0",
    "arrival_rate": "0.075",
    "mean_value": "100000",
    "kappa": "0.05",
    "shock_var": "150000000",
    "private_value_var": "100000000",
  }
}
\end{verbatim}

An explanation of select parameters follows:
\begin{itemize}
\item \verb|assignment|: Assignment of strategies to players. Relevant only in EGTA use case. This section must be \emph{empty} if simulating multiple market models.
%If there are multiple market models in the simulation, then all players present in \emph{any} model must have a strategy assigned in this section.

\begin{itemize}
\item Currently two types of roles are allowed: \verb|MARKETMAKER| and \verb|BACKGROUND|.
\item Each player's strategy is specified with parameter-value pairs in the format \\
\verb|<param1>_<val1>_...<paramN>_<valN>|
\end{itemize}


\item \verb|configuration|: Specify simulation configuration.

\item \verb|sim_length|: Length of simulation (in milliseconds).
\item \verb|tick_size|: Tick size. Prices are integers, so the smallest tick size is $1$.
\item \verb|primary_model|: For EGTA use case only. Specifies the market model that will have its players' payoff recorded separately.
\item \verb|[MARKETMODEL]|: Specifies number and configuration of market models. Possible market models include \verb|TWOMARKET|, \verb|CENTRALCDA|, \verb|CENTRALCALL|.

\begin{itemize}
\item Each model may have various configurations specifying, for example, the agents allowed in that instance of the model.

\item Each set of configurations is a comma-separated string in the specification file.

\item \verb|"MARKETMODEL": "A,B"| would indicate that for the given model, there is one instance of configuration A and one instance of configuration B. The system, in this case, would create two instances of this model (with differing configurations).

\item For example, \verb|"TWOMARKET": "LA,DUMMY"| creates a two-market model with LA and a two-market model without LA. 
\end{itemize}

\item \verb|[ENVIRONMENT AGENT TYPE]|: Specifies total number of each type of environment, or background agent (\verb|BackgroundAgent| class) within each given market model. Possible background agents include: \verb|BASICMM| (a basic market maker), \verb|ZI|, \verb|ZIP|, and \verb|ZIR| (ZI agent with re-entry).

\begin{itemize}
\item Environment agents are initialized with common random numbers across all market models.

\item If there are multiple markets, the system will distribute these background agents as evenly as possible amongst all markets in the model.

\item Note that environment agents can be specified in both use cases.

\end{itemize}

\item \verb|[ENVIRONMENT AGENT SETUP]|: Specifies parameters of each type of environment agent. Possible background agents include: \verb|BASICMM| (a basic market maker), \verb|ZI|, \verb|ZIP|, and \verb|ZIR| (ZI agent with re-entry).

\begin{itemize}
\item Specify setup with parameter-value pairs in the format \\
\verb|<param1>_<val1>_...<paramN>_<valN>|
\end{itemize}

\end{itemize}

% ---------------------
\section{Example: Running a simulation}

To run a basic simulation, first compile the code (run \verb|ant| in the base directory that contains the \verb|build.xml| file. Then use the following command:
\begin{verbatim}
./example.sh <HFT type> <NBBO latency> <number of samples> <CSV>
\end{verbatim}
For example:
\begin{verbatim}
./example.sh LA 0 10 test.csv
\end{verbatim}



All simulations should be saved within subfolders in the \verb|simulations| directory.
%The \verb|example.sh| script saves all observations in \verb|[HFT type]\_latency\_[NBBO latency]|.
In the previous example, the \verb|example.sh| script will generate a folder with path \verb|simulations\LA_latency_0|.
%
The resulting simulations will be parsed and saved in a CSV file in the base directory.

\end{document}


% ---------------------
\section{Log file}

Note that within the \verb|config| directory there is a file called \verb|env.properties|, which is used to specify the logging level. Log level $2$ will be adequate for most debugging purposes.

A table of frequently encountered symbols and \hl{XXX} in the log files. While printing to \verb|System.out| will be useful for initial debugging, going through the logs and observation files will be necessary in order to verify that agents are behaving as expected.


%TODO pop-out boxes in LaTeX?

Reading log files

Log IDs versus agent IDs
In general, all agents will have positive IDs, while Markets have negative IDs.



% ---------------------
\section{Observation file}


% ---------------------
\section{Data analysis}






